% Options for packages loaded elsewhere
\PassOptionsToPackage{unicode}{hyperref}
\PassOptionsToPackage{hyphens}{url}
%
\documentclass[
]{article}
\usepackage{lmodern}
\usepackage{amsmath}
\usepackage{ifxetex,ifluatex}
\ifnum 0\ifxetex 1\fi\ifluatex 1\fi=0 % if pdftex
  \usepackage[T1]{fontenc}
  \usepackage[utf8]{inputenc}
  \usepackage{textcomp} % provide euro and other symbols
  \usepackage{amssymb}
\else % if luatex or xetex
  \usepackage{unicode-math}
  \defaultfontfeatures{Scale=MatchLowercase}
  \defaultfontfeatures[\rmfamily]{Ligatures=TeX,Scale=1}
\fi
% Use upquote if available, for straight quotes in verbatim environments
\IfFileExists{upquote.sty}{\usepackage{upquote}}{}
\IfFileExists{microtype.sty}{% use microtype if available
  \usepackage[]{microtype}
  \UseMicrotypeSet[protrusion]{basicmath} % disable protrusion for tt fonts
}{}
\makeatletter
\@ifundefined{KOMAClassName}{% if non-KOMA class
  \IfFileExists{parskip.sty}{%
    \usepackage{parskip}
  }{% else
    \setlength{\parindent}{0pt}
    \setlength{\parskip}{6pt plus 2pt minus 1pt}}
}{% if KOMA class
  \KOMAoptions{parskip=half}}
\makeatother
\usepackage{xcolor}
\IfFileExists{xurl.sty}{\usepackage{xurl}}{} % add URL line breaks if available
\IfFileExists{bookmark.sty}{\usepackage{bookmark}}{\usepackage{hyperref}}
\hypersetup{
  pdftitle={Vaccini Notebook},
  hidelinks,
  pdfcreator={LaTeX via pandoc}}
\urlstyle{same} % disable monospaced font for URLs
\usepackage[margin=1in]{geometry}
\usepackage{color}
\usepackage{fancyvrb}
\newcommand{\VerbBar}{|}
\newcommand{\VERB}{\Verb[commandchars=\\\{\}]}
\DefineVerbatimEnvironment{Highlighting}{Verbatim}{commandchars=\\\{\}}
% Add ',fontsize=\small' for more characters per line
\usepackage{framed}
\definecolor{shadecolor}{RGB}{248,248,248}
\newenvironment{Shaded}{\begin{snugshade}}{\end{snugshade}}
\newcommand{\AlertTok}[1]{\textcolor[rgb]{0.94,0.16,0.16}{#1}}
\newcommand{\AnnotationTok}[1]{\textcolor[rgb]{0.56,0.35,0.01}{\textbf{\textit{#1}}}}
\newcommand{\AttributeTok}[1]{\textcolor[rgb]{0.77,0.63,0.00}{#1}}
\newcommand{\BaseNTok}[1]{\textcolor[rgb]{0.00,0.00,0.81}{#1}}
\newcommand{\BuiltInTok}[1]{#1}
\newcommand{\CharTok}[1]{\textcolor[rgb]{0.31,0.60,0.02}{#1}}
\newcommand{\CommentTok}[1]{\textcolor[rgb]{0.56,0.35,0.01}{\textit{#1}}}
\newcommand{\CommentVarTok}[1]{\textcolor[rgb]{0.56,0.35,0.01}{\textbf{\textit{#1}}}}
\newcommand{\ConstantTok}[1]{\textcolor[rgb]{0.00,0.00,0.00}{#1}}
\newcommand{\ControlFlowTok}[1]{\textcolor[rgb]{0.13,0.29,0.53}{\textbf{#1}}}
\newcommand{\DataTypeTok}[1]{\textcolor[rgb]{0.13,0.29,0.53}{#1}}
\newcommand{\DecValTok}[1]{\textcolor[rgb]{0.00,0.00,0.81}{#1}}
\newcommand{\DocumentationTok}[1]{\textcolor[rgb]{0.56,0.35,0.01}{\textbf{\textit{#1}}}}
\newcommand{\ErrorTok}[1]{\textcolor[rgb]{0.64,0.00,0.00}{\textbf{#1}}}
\newcommand{\ExtensionTok}[1]{#1}
\newcommand{\FloatTok}[1]{\textcolor[rgb]{0.00,0.00,0.81}{#1}}
\newcommand{\FunctionTok}[1]{\textcolor[rgb]{0.00,0.00,0.00}{#1}}
\newcommand{\ImportTok}[1]{#1}
\newcommand{\InformationTok}[1]{\textcolor[rgb]{0.56,0.35,0.01}{\textbf{\textit{#1}}}}
\newcommand{\KeywordTok}[1]{\textcolor[rgb]{0.13,0.29,0.53}{\textbf{#1}}}
\newcommand{\NormalTok}[1]{#1}
\newcommand{\OperatorTok}[1]{\textcolor[rgb]{0.81,0.36,0.00}{\textbf{#1}}}
\newcommand{\OtherTok}[1]{\textcolor[rgb]{0.56,0.35,0.01}{#1}}
\newcommand{\PreprocessorTok}[1]{\textcolor[rgb]{0.56,0.35,0.01}{\textit{#1}}}
\newcommand{\RegionMarkerTok}[1]{#1}
\newcommand{\SpecialCharTok}[1]{\textcolor[rgb]{0.00,0.00,0.00}{#1}}
\newcommand{\SpecialStringTok}[1]{\textcolor[rgb]{0.31,0.60,0.02}{#1}}
\newcommand{\StringTok}[1]{\textcolor[rgb]{0.31,0.60,0.02}{#1}}
\newcommand{\VariableTok}[1]{\textcolor[rgb]{0.00,0.00,0.00}{#1}}
\newcommand{\VerbatimStringTok}[1]{\textcolor[rgb]{0.31,0.60,0.02}{#1}}
\newcommand{\WarningTok}[1]{\textcolor[rgb]{0.56,0.35,0.01}{\textbf{\textit{#1}}}}
\usepackage{graphicx}
\makeatletter
\def\maxwidth{\ifdim\Gin@nat@width>\linewidth\linewidth\else\Gin@nat@width\fi}
\def\maxheight{\ifdim\Gin@nat@height>\textheight\textheight\else\Gin@nat@height\fi}
\makeatother
% Scale images if necessary, so that they will not overflow the page
% margins by default, and it is still possible to overwrite the defaults
% using explicit options in \includegraphics[width, height, ...]{}
\setkeys{Gin}{width=\maxwidth,height=\maxheight,keepaspectratio}
% Set default figure placement to htbp
\makeatletter
\def\fps@figure{htbp}
\makeatother
\setlength{\emergencystretch}{3em} % prevent overfull lines
\providecommand{\tightlist}{%
  \setlength{\itemsep}{0pt}\setlength{\parskip}{0pt}}
\setcounter{secnumdepth}{-\maxdimen} % remove section numbering
\ifluatex
  \usepackage{selnolig}  % disable illegal ligatures
\fi

\title{Vaccini Notebook}
\author{}
\date{\vspace{-2.5em}}

\begin{document}
\maketitle

\begin{Shaded}
\begin{Highlighting}[]
\FunctionTok{library}\NormalTok{(tidyverse)}
\end{Highlighting}
\end{Shaded}

\hypertarget{import-dati}{%
\section{Import dati}\label{import-dati}}

Da piattaforma open data del governo.

Dati somministrazioni giornaliere

\begin{Shaded}
\begin{Highlighting}[]
\FunctionTok{library}\NormalTok{(readr)}
\NormalTok{somministrazioni\_vaccini }\OtherTok{\textless{}{-}} \FunctionTok{read\_csv}\NormalTok{(}\StringTok{"https://raw.githubusercontent.com/italia/covid19{-}opendata{-}vaccini/master/dati/somministrazioni{-}vaccini{-}latest.csv"}\NormalTok{, }
    \AttributeTok{col\_types =} \FunctionTok{cols}\NormalTok{(}\AttributeTok{data\_somministrazione =} \FunctionTok{col\_date}\NormalTok{(}\AttributeTok{format =} \StringTok{"\%Y{-}\%m{-}\%d"}\NormalTok{), }
        \AttributeTok{sesso\_maschile =} \FunctionTok{col\_integer}\NormalTok{(), }
        \AttributeTok{sesso\_femminile =} \FunctionTok{col\_integer}\NormalTok{(), }
        \AttributeTok{prima\_dose =} \FunctionTok{col\_integer}\NormalTok{(), }\AttributeTok{seconda\_dose =} \FunctionTok{col\_integer}\NormalTok{(), }
        \AttributeTok{codice\_NUTS1 =} \FunctionTok{col\_skip}\NormalTok{(), }\AttributeTok{codice\_NUTS2 =} \FunctionTok{col\_skip}\NormalTok{(), }
        \AttributeTok{codice\_regione\_ISTAT =} \FunctionTok{col\_skip}\NormalTok{()))}

\FunctionTok{head}\NormalTok{(somministrazioni\_vaccini)}
\end{Highlighting}
\end{Shaded}

\begin{verbatim}
## # A tibble: 6 x 9
##   data_somministr~ fornitore area  fascia_anagrafi~ sesso_maschile
##   <date>           <chr>     <chr> <chr>                     <int>
## 1 2020-12-27       Pfizer/B~ ABR   20-29                         1
## 2 2020-12-27       Pfizer/B~ ABR   30-39                         1
## 3 2020-12-27       Pfizer/B~ ABR   40-49                         1
## 4 2020-12-27       Pfizer/B~ ABR   50-59                         4
## 5 2020-12-27       Pfizer/B~ ABR   60-69                        10
## 6 2020-12-27       Pfizer/B~ ABR   70-79                         1
## # ... with 4 more variables: sesso_femminile <int>, prima_dose <int>,
## #   seconda_dose <int>, nome_area <chr>
\end{verbatim}

Dati sulla platea totale di riceventi

\begin{Shaded}
\begin{Highlighting}[]
\NormalTok{platea }\OtherTok{\textless{}{-}} \FunctionTok{read\_csv}\NormalTok{(}\StringTok{"https://raw.githubusercontent.com/italia/covid19{-}opendata{-}vaccini/master/dati/platea.csv"}\NormalTok{, }
    \AttributeTok{col\_types =} \FunctionTok{cols}\NormalTok{(}\AttributeTok{totale\_popolazione =} \FunctionTok{col\_integer}\NormalTok{()))}

\FunctionTok{head}\NormalTok{(platea)}
\end{Highlighting}
\end{Shaded}

\begin{verbatim}
## # A tibble: 6 x 4
##   area  nome_area fascia_anagrafica totale_popolazione
##   <chr> <chr>     <chr>                          <int>
## 1 ABR   Abruzzo   12-19                          94727
## 2 ABR   Abruzzo   20-29                         125230
## 3 ABR   Abruzzo   30-39                         146965
## 4 ABR   Abruzzo   40-49                         187162
## 5 ABR   Abruzzo   50-59                         208686
## 6 ABR   Abruzzo   60-69                         171793
\end{verbatim}

\hypertarget{preparazione-dei-dati}{%
\section{Preparazione dei dati}\label{preparazione-dei-dati}}

\begin{Shaded}
\begin{Highlighting}[]
\NormalTok{platea }\OtherTok{\textless{}{-}}\NormalTok{ platea }\SpecialCharTok{\%\textgreater{}\%} 
  \FunctionTok{mutate}\NormalTok{( }\AttributeTok{area =} \FunctionTok{as.factor}\NormalTok{(area),}
          \AttributeTok{fascia\_anagrafica =} \FunctionTok{as.factor}\NormalTok{(fascia\_anagrafica),}
\NormalTok{          )}

\NormalTok{somministrazioni\_vaccini}\SpecialCharTok{$}\NormalTok{fascia\_anagrafica }\OtherTok{\textless{}{-}} 
  \FunctionTok{replace}\NormalTok{( somministrazioni\_vaccini}\SpecialCharTok{$}\NormalTok{fascia\_anagrafica, }
\NormalTok{           somministrazioni\_vaccini}\SpecialCharTok{$}\NormalTok{fascia\_anagrafica }\SpecialCharTok{\%in\%} \FunctionTok{c}\NormalTok{(}\StringTok{"80{-}89"}\NormalTok{, }\StringTok{"90+"}\NormalTok{), }
           \StringTok{"80+"}\NormalTok{)}
  
\NormalTok{somministrazioni\_vaccini }\OtherTok{\textless{}{-}}\NormalTok{ somministrazioni\_vaccini }\SpecialCharTok{\%\textgreater{}\%} 
  \FunctionTok{mutate}\NormalTok{(}\AttributeTok{fornitore =} \FunctionTok{as.factor}\NormalTok{(fornitore),}
         \AttributeTok{area =} \FunctionTok{as.factor}\NormalTok{(area),}
         \AttributeTok{fascia\_anagrafica =} \FunctionTok{as.factor}\NormalTok{(fascia\_anagrafica),}
\NormalTok{         ) }\SpecialCharTok{\%\textgreater{}\%} 
  \FunctionTok{select}\NormalTok{( }\SpecialCharTok{{-}}\FunctionTok{c}\NormalTok{(sesso\_maschile, sesso\_femminile))}
\end{Highlighting}
\end{Shaded}

Unione somministrazioni e platea (tramite codice regione)

\begin{Shaded}
\begin{Highlighting}[]
\NormalTok{vaccini }\OtherTok{\textless{}{-}} \FunctionTok{left\_join}\NormalTok{(somministrazioni\_vaccini, platea, }
                     \AttributeTok{by =} \FunctionTok{c}\NormalTok{(}\StringTok{"area"}\NormalTok{, }\StringTok{"fascia\_anagrafica"}\NormalTok{))}
\end{Highlighting}
\end{Shaded}

\hypertarget{analisi-prima-dose---italia}{%
\section{Analisi prima dose -
Italia}\label{analisi-prima-dose---italia}}

\begin{Shaded}
\begin{Highlighting}[]
\FunctionTok{ggplot}\NormalTok{(vaccini, }
       \FunctionTok{aes}\NormalTok{( }\AttributeTok{x =}\NormalTok{ data\_somministrazione,}
            \AttributeTok{y =}\NormalTok{ prima\_dose,}
            \AttributeTok{color =}\NormalTok{ fascia\_anagrafica)) }\SpecialCharTok{+}
  \FunctionTok{geom\_line}\NormalTok{() }\SpecialCharTok{+}
  \FunctionTok{facet\_wrap}\NormalTok{( }\SpecialCharTok{\textasciitilde{}}\NormalTok{fascia\_anagrafica)}
\end{Highlighting}
\end{Shaded}

\includegraphics{EDA_files/figure-latex/unnamed-chunk-6-1.pdf}

\begin{Shaded}
\begin{Highlighting}[]
\NormalTok{platea\_summary }\OtherTok{\textless{}{-}}\NormalTok{ platea }\SpecialCharTok{\%\textgreater{}\%} 
  \FunctionTok{group\_by}\NormalTok{(fascia\_anagrafica) }\SpecialCharTok{\%\textgreater{}\%} 
  \FunctionTok{summarise}\NormalTok{(}\AttributeTok{totale\_popolazione =} \FunctionTok{sum}\NormalTok{(totale\_popolazione))}

\NormalTok{prog\_vaccini }\OtherTok{\textless{}{-}}\NormalTok{ vaccini }\SpecialCharTok{\%\textgreater{}\%}
  \FunctionTok{group\_by}\NormalTok{(data\_somministrazione, fascia\_anagrafica) }\SpecialCharTok{\%\textgreater{}\%} 
  \FunctionTok{summarise}\NormalTok{(}\AttributeTok{prima\_dose =} \FunctionTok{sum}\NormalTok{(prima\_dose)) }\SpecialCharTok{\%\textgreater{}\%} 
  \FunctionTok{ungroup}\NormalTok{() }\SpecialCharTok{\%\textgreater{}\%} 
  \FunctionTok{group\_by}\NormalTok{(fascia\_anagrafica) }\SpecialCharTok{\%\textgreater{}\%} 
  \FunctionTok{mutate}\NormalTok{( }\AttributeTok{prog\_shot =} \FunctionTok{cumsum}\NormalTok{(prima\_dose)) }\SpecialCharTok{\%\textgreater{}\%} 
  \FunctionTok{left\_join}\NormalTok{(platea\_summary) }\SpecialCharTok{\%\textgreater{}\%} 
  \FunctionTok{mutate}\NormalTok{( }\AttributeTok{prog\_perc =}\NormalTok{ prog\_shot }\SpecialCharTok{/}\NormalTok{ totale\_popolazione) }\SpecialCharTok{\%\textgreater{}\%} 
  \FunctionTok{ungroup}\NormalTok{()}
\end{Highlighting}
\end{Shaded}

\begin{Shaded}
\begin{Highlighting}[]
\FunctionTok{ggplot}\NormalTok{(prog\_vaccini, }
       \FunctionTok{aes}\NormalTok{( }\AttributeTok{x =}\NormalTok{ data\_somministrazione,}
            \AttributeTok{y =}\NormalTok{ prog\_perc}\SpecialCharTok{*}\DecValTok{100}\NormalTok{,}
            \AttributeTok{color =}\NormalTok{ fascia\_anagrafica)) }\SpecialCharTok{+}
  \FunctionTok{geom\_line}\NormalTok{(}\AttributeTok{size=}\DecValTok{1}\NormalTok{) }\SpecialCharTok{+}
  \FunctionTok{coord\_cartesian}\NormalTok{(}\AttributeTok{ylim =} \FunctionTok{c}\NormalTok{(}\DecValTok{0}\NormalTok{,}\DecValTok{100}\NormalTok{)) }\SpecialCharTok{+}
  \FunctionTok{labs}\NormalTok{( }\AttributeTok{title =} \StringTok{"Vaccinati prima dose"}\NormalTok{,}
        \AttributeTok{subtitle =} \StringTok{"Percentuale di completamento per età"}\NormalTok{,}
        \AttributeTok{y =} \ConstantTok{NULL}\NormalTok{,}
        \AttributeTok{x =} \ConstantTok{NULL}\NormalTok{)}
\end{Highlighting}
\end{Shaded}

\includegraphics{EDA_files/figure-latex/unnamed-chunk-8-1.pdf}

Ogni curva ha tre fasi: la prima che sale lentamente, una fase ripida
dove molti di vaccinano in poco tempo, una terza parte più piatta che
corrisponde alle difficoltà nel coinvolgere la porzione più esitante per
quella fascia d'età.

In altre parole, più la curva è ripida più la gente è rapida
nell'aderire alle opportunità di vaccinarsi per la loro fascia d'età:

\begin{itemize}
\tightlist
\item
  gli 80+ sono quasi completamente vaccinati
\item
  i 70-79 si sono fatti vaccinare rapidamente
\item
  andando verso i più giovani si nota una progressivo appiattimento
  della fase ripida
\end{itemize}

\hypertarget{analisi-prima-dose---regioni}{%
\section{Analisi prima dose -
Regioni}\label{analisi-prima-dose---regioni}}

\begin{Shaded}
\begin{Highlighting}[]
\NormalTok{reg\_vaccini }\OtherTok{\textless{}{-}}\NormalTok{ vaccini }\SpecialCharTok{\%\textgreater{}\%}
  \FunctionTok{group\_by}\NormalTok{(area, data\_somministrazione, fascia\_anagrafica) }\SpecialCharTok{\%\textgreater{}\%} 
  \FunctionTok{summarise}\NormalTok{(}\AttributeTok{prima\_dose =} \FunctionTok{sum}\NormalTok{(prima\_dose)) }\SpecialCharTok{\%\textgreater{}\%} 
  \FunctionTok{ungroup}\NormalTok{() }\SpecialCharTok{\%\textgreater{}\%} 
  \FunctionTok{group\_by}\NormalTok{(area, fascia\_anagrafica) }\SpecialCharTok{\%\textgreater{}\%} 
  \FunctionTok{mutate}\NormalTok{( }\AttributeTok{prog\_shot =} \FunctionTok{cumsum}\NormalTok{(prima\_dose)) }\SpecialCharTok{\%\textgreater{}\%} 
  \FunctionTok{left\_join}\NormalTok{(platea) }\SpecialCharTok{\%\textgreater{}\%} 
  \FunctionTok{mutate}\NormalTok{( }\AttributeTok{prog\_perc =}\NormalTok{ prog\_shot }\SpecialCharTok{/}\NormalTok{ totale\_popolazione) }\SpecialCharTok{\%\textgreater{}\%} 
  \FunctionTok{ungroup}\NormalTok{()}
\end{Highlighting}
\end{Shaded}

\includegraphics{EDA_files/figure-latex/unnamed-chunk-10-1.pdf}

\end{document}
