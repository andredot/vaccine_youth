\documentclass[]{article}
\usepackage{lmodern}
\usepackage{amssymb,amsmath}
\usepackage{ifxetex,ifluatex}
\usepackage{fixltx2e} % provides \textsubscript
\ifnum 0\ifxetex 1\fi\ifluatex 1\fi=0 % if pdftex
  \usepackage[T1]{fontenc}
  \usepackage[utf8]{inputenc}
\else % if luatex or xelatex
  \ifxetex
    \usepackage{mathspec}
  \else
    \usepackage{fontspec}
  \fi
  \defaultfontfeatures{Ligatures=TeX,Scale=MatchLowercase}
\fi
% use upquote if available, for straight quotes in verbatim environments
\IfFileExists{upquote.sty}{\usepackage{upquote}}{}
% use microtype if available
\IfFileExists{microtype.sty}{%
\usepackage[]{microtype}
\UseMicrotypeSet[protrusion]{basicmath} % disable protrusion for tt fonts
}{}
\PassOptionsToPackage{hyphens}{url} % url is loaded by hyperref
\usepackage[unicode=true]{hyperref}
\hypersetup{
            pdftitle={Vaccini Notebook},
            pdfborder={0 0 0},
            breaklinks=true}
\urlstyle{same}  % don't use monospace font for urls
\usepackage[margin=1in]{geometry}
\usepackage{color}
\usepackage{fancyvrb}
\newcommand{\VerbBar}{|}
\newcommand{\VERB}{\Verb[commandchars=\\\{\}]}
\DefineVerbatimEnvironment{Highlighting}{Verbatim}{commandchars=\\\{\}}
% Add ',fontsize=\small' for more characters per line
\usepackage{framed}
\definecolor{shadecolor}{RGB}{248,248,248}
\newenvironment{Shaded}{\begin{snugshade}}{\end{snugshade}}
\newcommand{\KeywordTok}[1]{\textcolor[rgb]{0.13,0.29,0.53}{\textbf{#1}}}
\newcommand{\DataTypeTok}[1]{\textcolor[rgb]{0.13,0.29,0.53}{#1}}
\newcommand{\DecValTok}[1]{\textcolor[rgb]{0.00,0.00,0.81}{#1}}
\newcommand{\BaseNTok}[1]{\textcolor[rgb]{0.00,0.00,0.81}{#1}}
\newcommand{\FloatTok}[1]{\textcolor[rgb]{0.00,0.00,0.81}{#1}}
\newcommand{\ConstantTok}[1]{\textcolor[rgb]{0.00,0.00,0.00}{#1}}
\newcommand{\CharTok}[1]{\textcolor[rgb]{0.31,0.60,0.02}{#1}}
\newcommand{\SpecialCharTok}[1]{\textcolor[rgb]{0.00,0.00,0.00}{#1}}
\newcommand{\StringTok}[1]{\textcolor[rgb]{0.31,0.60,0.02}{#1}}
\newcommand{\VerbatimStringTok}[1]{\textcolor[rgb]{0.31,0.60,0.02}{#1}}
\newcommand{\SpecialStringTok}[1]{\textcolor[rgb]{0.31,0.60,0.02}{#1}}
\newcommand{\ImportTok}[1]{#1}
\newcommand{\CommentTok}[1]{\textcolor[rgb]{0.56,0.35,0.01}{\textit{#1}}}
\newcommand{\DocumentationTok}[1]{\textcolor[rgb]{0.56,0.35,0.01}{\textbf{\textit{#1}}}}
\newcommand{\AnnotationTok}[1]{\textcolor[rgb]{0.56,0.35,0.01}{\textbf{\textit{#1}}}}
\newcommand{\CommentVarTok}[1]{\textcolor[rgb]{0.56,0.35,0.01}{\textbf{\textit{#1}}}}
\newcommand{\OtherTok}[1]{\textcolor[rgb]{0.56,0.35,0.01}{#1}}
\newcommand{\FunctionTok}[1]{\textcolor[rgb]{0.00,0.00,0.00}{#1}}
\newcommand{\VariableTok}[1]{\textcolor[rgb]{0.00,0.00,0.00}{#1}}
\newcommand{\ControlFlowTok}[1]{\textcolor[rgb]{0.13,0.29,0.53}{\textbf{#1}}}
\newcommand{\OperatorTok}[1]{\textcolor[rgb]{0.81,0.36,0.00}{\textbf{#1}}}
\newcommand{\BuiltInTok}[1]{#1}
\newcommand{\ExtensionTok}[1]{#1}
\newcommand{\PreprocessorTok}[1]{\textcolor[rgb]{0.56,0.35,0.01}{\textit{#1}}}
\newcommand{\AttributeTok}[1]{\textcolor[rgb]{0.77,0.63,0.00}{#1}}
\newcommand{\RegionMarkerTok}[1]{#1}
\newcommand{\InformationTok}[1]{\textcolor[rgb]{0.56,0.35,0.01}{\textbf{\textit{#1}}}}
\newcommand{\WarningTok}[1]{\textcolor[rgb]{0.56,0.35,0.01}{\textbf{\textit{#1}}}}
\newcommand{\AlertTok}[1]{\textcolor[rgb]{0.94,0.16,0.16}{#1}}
\newcommand{\ErrorTok}[1]{\textcolor[rgb]{0.64,0.00,0.00}{\textbf{#1}}}
\newcommand{\NormalTok}[1]{#1}
\usepackage{graphicx,grffile}
\makeatletter
\def\maxwidth{\ifdim\Gin@nat@width>\linewidth\linewidth\else\Gin@nat@width\fi}
\def\maxheight{\ifdim\Gin@nat@height>\textheight\textheight\else\Gin@nat@height\fi}
\makeatother
% Scale images if necessary, so that they will not overflow the page
% margins by default, and it is still possible to overwrite the defaults
% using explicit options in \includegraphics[width, height, ...]{}
\setkeys{Gin}{width=\maxwidth,height=\maxheight,keepaspectratio}
\IfFileExists{parskip.sty}{%
\usepackage{parskip}
}{% else
\setlength{\parindent}{0pt}
\setlength{\parskip}{6pt plus 2pt minus 1pt}
}
\setlength{\emergencystretch}{3em}  % prevent overfull lines
\providecommand{\tightlist}{%
  \setlength{\itemsep}{0pt}\setlength{\parskip}{0pt}}
\setcounter{secnumdepth}{0}
% Redefines (sub)paragraphs to behave more like sections
\ifx\paragraph\undefined\else
\let\oldparagraph\paragraph
\renewcommand{\paragraph}[1]{\oldparagraph{#1}\mbox{}}
\fi
\ifx\subparagraph\undefined\else
\let\oldsubparagraph\subparagraph
\renewcommand{\subparagraph}[1]{\oldsubparagraph{#1}\mbox{}}
\fi

% set default figure placement to htbp
\makeatletter
\def\fps@figure{htbp}
\makeatother


\title{Vaccini Notebook}
\author{}
\date{\vspace{-2.5em}}

\begin{document}
\maketitle

\begin{Shaded}
\begin{Highlighting}[]
\KeywordTok{library}\NormalTok{(tidyverse)}
\end{Highlighting}
\end{Shaded}

\section{Import dati}\label{import-dati}

Da piattaforma open data del governo.

Dati somministrazioni giornaliere

\begin{Shaded}
\begin{Highlighting}[]
\KeywordTok{library}\NormalTok{(readr)}
\NormalTok{somministrazioni_vaccini <-}\StringTok{ }\KeywordTok{read_csv}\NormalTok{(}\StringTok{"https://raw.githubusercontent.com/italia/covid19-opendata-vaccini/master/dati/somministrazioni-vaccini-latest.csv"}\NormalTok{, }
    \DataTypeTok{col_types =} \KeywordTok{cols}\NormalTok{(}\DataTypeTok{data_somministrazione =} \KeywordTok{col_date}\NormalTok{(}\DataTypeTok{format =} \StringTok{"%Y-%m-%d"}\NormalTok{), }
        \DataTypeTok{sesso_maschile =} \KeywordTok{col_integer}\NormalTok{(), }
        \DataTypeTok{sesso_femminile =} \KeywordTok{col_integer}\NormalTok{(), }
        \DataTypeTok{prima_dose =} \KeywordTok{col_integer}\NormalTok{(), }\DataTypeTok{seconda_dose =} \KeywordTok{col_integer}\NormalTok{(), }
        \DataTypeTok{codice_NUTS1 =} \KeywordTok{col_skip}\NormalTok{(), }\DataTypeTok{codice_NUTS2 =} \KeywordTok{col_skip}\NormalTok{(), }
        \DataTypeTok{codice_regione_ISTAT =} \KeywordTok{col_skip}\NormalTok{()))}

\KeywordTok{head}\NormalTok{(somministrazioni_vaccini)}
\end{Highlighting}
\end{Shaded}

\begin{verbatim}
## # A tibble: 6 x 11
##   data_somministrazione fornitore       area  fascia_anagrafica sesso_maschile
##   <date>                <chr>           <chr> <chr>                      <int>
## 1 2020-12-27            Pfizer/BioNTech ABR   20-29                          1
## 2 2020-12-27            Pfizer/BioNTech ABR   30-39                          1
## 3 2020-12-27            Pfizer/BioNTech ABR   40-49                          1
## 4 2020-12-27            Pfizer/BioNTech ABR   50-59                          7
## 5 2020-12-27            Pfizer/BioNTech ABR   60-69                         12
## 6 2020-12-27            Pfizer/BioNTech ABR   70-79                          1
## # ... with 6 more variables: sesso_femminile <int>, prima_dose <int>,
## #   seconda_dose <int>, pregressa_infezione <dbl>,
## #   dose_addizionale_booster <dbl>, nome_area <chr>
\end{verbatim}

Dati sulla platea totale di riceventi

\begin{Shaded}
\begin{Highlighting}[]
\NormalTok{platea <-}\StringTok{ }\KeywordTok{read_csv}\NormalTok{(}\StringTok{"https://raw.githubusercontent.com/italia/covid19-opendata-vaccini/master/dati/platea.csv"}\NormalTok{, }
    \DataTypeTok{col_types =} \KeywordTok{cols}\NormalTok{(}\DataTypeTok{totale_popolazione =} \KeywordTok{col_integer}\NormalTok{()))}

\KeywordTok{head}\NormalTok{(platea)}
\end{Highlighting}
\end{Shaded}

\begin{verbatim}
## # A tibble: 6 x 4
##   area  nome_area fascia_anagrafica totale_popolazione
##   <chr> <chr>     <chr>                          <int>
## 1 ABR   Abruzzo   05-11                          76431
## 2 ABR   Abruzzo   12-19                          94727
## 3 ABR   Abruzzo   20-29                         125230
## 4 ABR   Abruzzo   30-39                         146965
## 5 ABR   Abruzzo   40-49                         187162
## 6 ABR   Abruzzo   50-59                         208686
\end{verbatim}

\section{Preparazione dei dati}\label{preparazione-dei-dati}

\begin{Shaded}
\begin{Highlighting}[]
\NormalTok{platea <-}\StringTok{ }\NormalTok{platea }\OperatorTok\StringTok{ }
\StringTok{  }\KeywordTok{mutate}\NormalTok{( }\DataTypeTok{area =} \KeywordTok{as.factor}\NormalTok{(area),}
          \DataTypeTok{fascia_anagrafica =} \KeywordTok{as.factor}\NormalTok{(fascia_anagrafica),}
\NormalTok{          )}

\NormalTok{somministrazioni_vaccini}\OperatorTok{$}\NormalTok{fascia_anagrafica <-}\StringTok{ }
\StringTok{  }\KeywordTok{replace}\NormalTok{( somministrazioni_vaccini}\OperatorTok{$}\NormalTok{fascia_anagrafica, }
\NormalTok{           somministrazioni_vaccini}\OperatorTok{$}\NormalTok{fascia_anagrafica }\OperatorTok\StringTok{ }\KeywordTok{c}\NormalTok{(}\StringTok{"80-89"}\NormalTok{, }\StringTok{"90+"}\NormalTok{), }
           \StringTok{"80+"}\NormalTok{)}
  
\NormalTok{somministrazioni_vaccini <-}\StringTok{ }\NormalTok{somministrazioni_vaccini }\OperatorTok\StringTok{ }
\StringTok{  }\KeywordTok{mutate}\NormalTok{(}\DataTypeTok{fornitore =} \KeywordTok{as.factor}\NormalTok{(fornitore),}
         \DataTypeTok{area =} \KeywordTok{as.factor}\NormalTok{(area),}
         \DataTypeTok{fascia_anagrafica =} \KeywordTok{as.factor}\NormalTok{(fascia_anagrafica),}
\NormalTok{         ) }\OperatorTok\StringTok{ }
\StringTok{  }\KeywordTok{select}\NormalTok{( }\OperatorTok{-}\KeywordTok{c}\NormalTok{(sesso_maschile, sesso_femminile))}
\end{Highlighting}
\end{Shaded}

Unione somministrazioni e platea (tramite codice regione)

\begin{Shaded}
\begin{Highlighting}[]
\NormalTok{vaccini <-}\StringTok{ }\KeywordTok{left_join}\NormalTok{(somministrazioni_vaccini, platea, }
                     \DataTypeTok{by =} \KeywordTok{c}\NormalTok{(}\StringTok{"area"}\NormalTok{, }\StringTok{"fascia_anagrafica"}\NormalTok{))}
\end{Highlighting}
\end{Shaded}

\section{Analisi prima dose - Italia}\label{analisi-prima-dose---italia}

\begin{Shaded}
\begin{Highlighting}[]
\KeywordTok{ggplot}\NormalTok{(vaccini, }
       \KeywordTok{aes}\NormalTok{( }\DataTypeTok{x =}\NormalTok{ data_somministrazione,}
            \DataTypeTok{y =}\NormalTok{ prima_dose,}
            \DataTypeTok{color =}\NormalTok{ fascia_anagrafica)) }\OperatorTok{+}
\StringTok{  }\KeywordTok{geom_line}\NormalTok{() }\OperatorTok{+}
\StringTok{  }\KeywordTok{facet_wrap}\NormalTok{( }\OperatorTok{~}\NormalTok{fascia_anagrafica)}
\end{Highlighting}
\end{Shaded}

\includegraphics{EDA_files/figure-latex/unnamed-chunk-6-1.pdf}

\begin{Shaded}
\begin{Highlighting}[]
\NormalTok{platea_summary <-}\StringTok{ }\NormalTok{platea }\OperatorTok\StringTok{ }
\StringTok{  }\KeywordTok{group_by}\NormalTok{(fascia_anagrafica) }\OperatorTok\StringTok{ }
\StringTok{  }\KeywordTok{summarise}\NormalTok{(}\DataTypeTok{totale_popolazione =} \KeywordTok{sum}\NormalTok{(totale_popolazione))}

\NormalTok{prog_vaccini <-}\StringTok{ }\NormalTok{vaccini }\OperatorTok
\StringTok{  }\KeywordTok{group_by}\NormalTok{(data_somministrazione, fascia_anagrafica) }\OperatorTok\StringTok{ }
\StringTok{  }\KeywordTok{summarise}\NormalTok{(}\DataTypeTok{prima_dose =} \KeywordTok{sum}\NormalTok{(prima_dose)) }\OperatorTok\StringTok{ }
\StringTok{  }\KeywordTok{ungroup}\NormalTok{() }\OperatorTok\StringTok{ }
\StringTok{  }\KeywordTok{group_by}\NormalTok{(fascia_anagrafica) }\OperatorTok\StringTok{ }
\StringTok{  }\KeywordTok{mutate}\NormalTok{( }\DataTypeTok{prog_shot =} \KeywordTok{cumsum}\NormalTok{(prima_dose)) }\OperatorTok\StringTok{ }
\StringTok{  }\KeywordTok{left_join}\NormalTok{(platea_summary) }\OperatorTok\StringTok{ }
\StringTok{  }\KeywordTok{mutate}\NormalTok{( }\DataTypeTok{prog_perc =}\NormalTok{ prog_shot }\OperatorTok{/}\StringTok{ }\NormalTok{totale_popolazione) }\OperatorTok\StringTok{ }
\StringTok{  }\KeywordTok{ungroup}\NormalTok{()}
\end{Highlighting}
\end{Shaded}

\begin{Shaded}
\begin{Highlighting}[]
\KeywordTok{ggplot}\NormalTok{(prog_vaccini, }
       \KeywordTok{aes}\NormalTok{( }\DataTypeTok{x =}\NormalTok{ data_somministrazione,}
            \DataTypeTok{y =}\NormalTok{ prog_perc}\OperatorTok{*}\DecValTok{100}\NormalTok{,}
            \DataTypeTok{color =}\NormalTok{ fascia_anagrafica)) }\OperatorTok{+}
\StringTok{  }\KeywordTok{geom_line}\NormalTok{(}\DataTypeTok{size=}\DecValTok{1}\NormalTok{) }\OperatorTok{+}
\StringTok{  }\KeywordTok{coord_cartesian}\NormalTok{(}\DataTypeTok{ylim =} \KeywordTok{c}\NormalTok{(}\DecValTok{0}\NormalTok{,}\DecValTok{100}\NormalTok{)) }\OperatorTok{+}
\StringTok{  }\KeywordTok{labs}\NormalTok{( }\DataTypeTok{title =} \StringTok{"Vaccinati prima dose"}\NormalTok{,}
        \DataTypeTok{subtitle =} \StringTok{"Percentuale di completamento per età"}\NormalTok{,}
        \DataTypeTok{y =} \OtherTok{NULL}\NormalTok{,}
        \DataTypeTok{x =} \OtherTok{NULL}\NormalTok{)}
\end{Highlighting}
\end{Shaded}

\includegraphics{EDA_files/figure-latex/unnamed-chunk-8-1.pdf}

Ogni curva ha tre fasi: la prima che sale lentamente, una fase ripida
dove molti di vaccinano in poco tempo, una terza parte più piatta che
corrisponde alle difficoltà nel coinvolgere la porzione più esitante per
quella fascia d'età.

In altre parole, più la curva è ripida più la gente è rapida
nell'aderire alle opportunità di vaccinarsi per la loro fascia d'età:

\begin{itemize}
\tightlist
\item
  gli 80+ sono quasi completamente vaccinati
\item
  i 70-79 si sono fatti vaccinare rapidamente
\item
  andando verso i più giovani si nota una progressivo appiattimento
  della fase ripida
\end{itemize}

\section{Analisi prima dose -
Regioni}\label{analisi-prima-dose---regioni}

\begin{Shaded}
\begin{Highlighting}[]
\NormalTok{reg_vaccini <-}\StringTok{ }\NormalTok{vaccini }\OperatorTok
\StringTok{  }\KeywordTok{group_by}\NormalTok{(area, data_somministrazione, fascia_anagrafica) }\OperatorTok\StringTok{ }
\StringTok{  }\KeywordTok{summarise}\NormalTok{(}\DataTypeTok{prima_dose =} \KeywordTok{sum}\NormalTok{(prima_dose)) }\OperatorTok\StringTok{ }
\StringTok{  }\KeywordTok{ungroup}\NormalTok{() }\OperatorTok\StringTok{ }
\StringTok{  }\KeywordTok{group_by}\NormalTok{(area, fascia_anagrafica) }\OperatorTok\StringTok{ }
\StringTok{  }\KeywordTok{mutate}\NormalTok{( }\DataTypeTok{prog_shot =} \KeywordTok{cumsum}\NormalTok{(prima_dose)) }\OperatorTok\StringTok{ }
\StringTok{  }\KeywordTok{left_join}\NormalTok{(platea) }\OperatorTok\StringTok{ }
\StringTok{  }\KeywordTok{mutate}\NormalTok{( }\DataTypeTok{prog_perc =}\NormalTok{ prog_shot }\OperatorTok{/}\StringTok{ }\NormalTok{totale_popolazione) }\OperatorTok\StringTok{ }
\StringTok{  }\KeywordTok{ungroup}\NormalTok{()}
\end{Highlighting}
\end{Shaded}

\includegraphics{EDA_files/figure-latex/unnamed-chunk-10-1.pdf}

Trentino

\begin{Shaded}
\begin{Highlighting}[]
\NormalTok{reg_vaccini }\OperatorTok\StringTok{ }
\StringTok{  }\KeywordTok{filter}\NormalTok{(area }\OperatorTok{==}\StringTok{ "PAT"}\NormalTok{) }\OperatorTok\StringTok{ }
\StringTok{  }\KeywordTok{ggplot}\NormalTok{(   }\KeywordTok{aes}\NormalTok{( }\DataTypeTok{x =}\NormalTok{ data_somministrazione,}
            \DataTypeTok{y =}\NormalTok{ prog_perc}\OperatorTok{*}\DecValTok{100}\NormalTok{,}
            \DataTypeTok{color =}\NormalTok{ fascia_anagrafica)) }\OperatorTok{+}
\StringTok{    }\KeywordTok{geom_line}\NormalTok{(}\DataTypeTok{size=}\DecValTok{1}\NormalTok{) }\OperatorTok{+}
\StringTok{    }\KeywordTok{coord_cartesian}\NormalTok{(}\DataTypeTok{ylim =} \KeywordTok{c}\NormalTok{(}\DecValTok{0}\NormalTok{,}\DecValTok{100}\NormalTok{)) }\OperatorTok{+}
\StringTok{    }\KeywordTok{labs}\NormalTok{( }\DataTypeTok{title =} \StringTok{"Vaccinati prima dose PAT"}\NormalTok{,}
        \DataTypeTok{subtitle =} \StringTok{"Percentuale di completamento per età"}\NormalTok{,}
        \DataTypeTok{y =} \OtherTok{NULL}\NormalTok{,}
        \DataTypeTok{x =} \OtherTok{NULL}\NormalTok{)}
\end{Highlighting}
\end{Shaded}

\includegraphics{EDA_files/figure-latex/unnamed-chunk-11-1.pdf}

\end{document}
